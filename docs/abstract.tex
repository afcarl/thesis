\begin{abstract}
  \doublespacenormalsize
  Statistical natural language processing (NLP) models assign a posterior distribution to the set of possible outcomes and make their predictions based on those scores. Current performance metrics for NLP systems only take into account the model final decisions, which not only depends on the quality of the system’s model but also the inference scheme, rather than explicitly reflect the quality of the prediction posterior distribution. In this thesis, I propose a metric for directly examining the quality of the posterior distribution, the calibration test. First of all, I will present the theoretical foundation for the concept of calibration. After that, I apply the test to numerous families of NLP models and show that the calibration is complementary to the traditional metrics in the sense that it provides a more comprehensive insights into the performances of NLP systems.

\end{abstract}
